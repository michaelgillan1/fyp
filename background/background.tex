\chapter{Background}

\section{Types and Type Systems}

Type systems are lightweight formal methods that allow a programmer to reason that a program will behave as intended with respect to a specification of the language it is created in. In his book \textit{Types and Programming Languages}\cite{PierceBenjaminC2002Tapl}, Pierce defines a type system:
\begin{displayquote}
A type system is a tractable syntactic method for proving the absence of certain program behaviours by classifying phrases according to the kinds of values they compute.
\end{displayquote}
From this definition, we can infer that types are the "kinds of values" that can result from computations within a program.

Traditionally, the concept of a \textit{record type} is used to define composite types made from more than one primitive type; a record, which can be thought of as an \textit{object} with its components representing the object's methods and fields, is an unordered set of labeled values, and a record type is specified by associating a type with each of its labels \cite{Cardelli1985}. In addition to composite and primitive types, we also have function types, which express the types of the parameters and the return type of a function.

Formally, a record consists of labels $x_1, ..., x_j$ with associated types $\sigma_1, ... \sigma_j$, and is represented as the set $\{x_1:\sigma_1, ..., x_j:\sigma_j\}$ \cite{Canning1989}.

\subsection{Structural Type Systems}

A structural type system is one in which type equivalence and compatibility is determined entirely by the type's structure; two types are compatible if for each feature in the first type, a corresponding and equivalent feature exists in the second type.

A language that employees this sort of type system is Go, when checking against interfaces. In the example below, \texttt{fmt.Println()} expects the parameter to implement the \texttt{fmt.Stringer} interface, which requires a method \texttt{String() string}. Since \texttt{User} implements this, it is accepted by \texttt{fmt.Println()}

\begin{verbatim}
  type User struct {
    name string
  }

  func (user User) String() string {
    return fmt.Sprintf("User: name = %s", user.name)
  }

  func main() {
    user := User{name: "foo"}
    fmt.Println(user)         // User: name = foo
  }
\end{verbatim}

\subsection{Nominal Type Systems}

A nominal type system is one in which type equivalence and compatibility is determined entirely by type declarations and names; two structurally identical types are not considered the same if they are declared with different names.

Nominal typing is used for checking interfaces in Java. An equivalent situation to the structural type example is given below.

\subsection{Union and Intersection Types}

Union and intersection types are forms of type expressions that allow the combination of types into more complex types, without the need for a definition.

\subsubsection{Union Types}

A union type describes values that belong to either of two types. A good example of this is \texttt{char} in C. A signed char generally has a range of -127 to 128, and an unsigned char has a range of 0 to 255. The union of a signed char and unsigned char could accept values in the range -127 to 255. Generally, valid operations on union types have to be valid of both of its component types.

\subsubsection{Intersection Types}

An intersection type describes values that belong to both of two types. Using the char example above, the intersection type of a signed and unsigned char would accept values in the range 0 to 128. This intersection type could then be used in any operation expecting either of its component types, because it is compatible with both.

\section{Subtyping}

A type S is a structural subtype of T if S’s interface (its public fields and methods) is a superset of T’s. Intuitively, any operation that we perform on T through its interface can also be performed on S.
Formally, we can use the concept of a record type, which describes the interface of an object as a mapping of labels to types. The record subtyping axiom is a formalization of the intuition above:

\begin{mathpar}
\inferrule{\sigma_1 \subseteq \rho_1, ..., \sigma_k \subseteq \rho_k}{\{x_1:\sigma_1, ...,  x_k:\sigma_k, ..., x_l:\sigma_l\} \subseteq \{x_1:\rho_1, ...,  x_k:\rho_k\} }
\end{mathpar}

Nominal subtyping is an explicitly defined subtype relation between two types. The below example from the Pony tutorial shows nominal subtyping in Pony; the relationship between ‘Bob’ and ‘Named’ is declared, and says that ‘Bob’ provides ‘Named’.
\begin{verbatim}
trait Named
  fun name(): String => "Bob"

class Bob is Named
\end{verbatim}
With nominal subtyping, it is the compiler’s responsibility to check that the subtype implements everything required by the subtyping relation.

\section{Covariance and Contravariance}

The concept of variance refers to how the subtyping between complex type expressions relates to the subtyping of the components of said expressions; the type constructor of complex types may preserve, reverse, or ignore the subtype relation between its component types.

A covariant type constructor is one that preserves the relationship between subtypes; if \texttt{foo} is a subtype of \texttt{bar}, and the type constructor of a \texttt{List} is covariant, then a \texttt{List<foo>} is a subtype of \texttt{List<bar>}.

A contravariant type constructor is one that reverses the relationship between subtypes; given the same subtype relationship between \texttt{foo} and \texttt{bar}, and a contravariant type constructor for a \texttt{Printer}, we have that \texttt{Printer<bar>} is a subtype of \texttt{Printer<foo>}.

Additionally, there is the concept of invariance, where the relationship between simple types is ignored in complex type constructors, and two different instances of a type constructor are neither a subtype or a supertype of one another.

When typing a function, the return type may be covariant when the function is overridden, and parameter types may be contravariant. Generally, wherever the base method's function may be called, the overridden function must be permitted.
Consider the psuedocode example below.

\begin{verbatim}
  class Foo extends Bar {}
  class Foo1 extends Bar1 {}

  class Thing {
    Bar doThing(Foo1 x) { 
      // implementation
    }
  }

  class OtherThing extends Thing {
    Foo doThing(Bar1 x) {
      // implementation
    }
  }
\end{verbatim}
First, it is clear that \texttt{OtherThing} can make \texttt{doThing} return a \texttt{Foo} as \texttt{Foo} is a subtype of \texttt{Bar}, and intuitively anything expecting a \texttt{Bar} can use a \texttt{Foo} instead. Secondly, \texttt{otherThing} can make \texttt{doThing} take a \texttt{Bar1} as a parameter since anything calling \texttt{Thing.doThing} must pass a \texttt{Foo1}, which is guaranteed to be a subtype of \texttt{Bar1}, and can be treated that as a \texttt{Bar1.}

The above leads to a standard rule for function subtyping (using the arrow for function typing):

\begin{mathpar}
\inferrule{\sigma' \subseteq \sigma \\ \tau \subseteq \tau'}{\sigma \to \tau \subseteq \sigma' \to \tau'}
\end{mathpar}

Some languages, such as Java, limit parameter types to be invariant, while others including C\# allow parameter types to be contravariant.

\section{Polymorphism} \label{polymorphism}

Polymorphism is defined as providing a single interface to entities of different types \cite{Stroustrup}. 

\subsection{Ad hoc Polymorphism}

Ad hoc polymorphism, commonly referred to as \textit{function overloading}, is the mechanism by which a function appears to work on multiple different types, even if these types do not share a similar structure \cite{Cardelli1985}. 

The common example for this is the \texttt{+} operator; when applied to integers, we have operations of the form \texttt{1 + 2 == 3}, where \texttt{+} is a summation function; when applied to strings, we can use it to perform concatenation, such as \texttt{"foo" + "bar" == "foobar"}. It is therefore clear that although this operator presents the same interface for many different types, it actually represents a number of distinct, heterogeneous functions.

\subsection{Subtype Polymorphism}

Since we know that subtypes present the same interface as their base type, polymorphism through subtyping allows a subtype to be used anywhere a supertype is expected. This is clear in the below example:
\begin{verbatim}
  class A
  class B
  class Pair
    var fst: Any tag
    var snd: Any tag
  
    new create(fst': Any tag, snd': Any tag) =>
      this.fst = consume fst'
      this.snd = consume snd'
    
    fun ref getfst() : Any tag =>
      this.fst

  actor Main
    new create(env: Env) =>
      var pairA : Pair ref = Pair.create(A.create(), A.create())
      var pairB : Pair ref = Pair.create(B.create(), B.create())
\end{verbatim}
Both \texttt{A} and \texttt{B} can be stored using the same \texttt{Pair} class. However, all type information is lost once the reference is stored. The following execution produces an error, as \texttt{getfst()} will only return an \texttt{Any}.
\begin{verbatim}
    actor Main
    new create(env: Env) =>
      var pairA : Pair ref = Pair.create(A.create(), A.create())
      
      \\ error: Any tag is not a subtype of A tag
      var a : A tag = pairA.getfst() 
\end{verbatim}

\subsection{Parametric Polymorphism}

Parametric polymorphism allows a single implementation of a function to work uniformly on many different types, which usually have some common structure. They are said to be \textit{generic} and accept type parameters as placeholders for types:

\begin{verbatim}
  class Pair[T]
    var fst: T 
    var snd: T 
  
    new create(fst': T, snd': T) =>
      this.fst = consume fst'
      this.snd = consume snd'
    
    fun ref getFst() : T =>
      this.fst

  actor Main
    new create(env: Env) =>
      var pairA : Pair[A ref] ref = 
        Pair[A ref].create(A.create(), A.create())
      var a: A ref = pairA.getFst()
\end{verbatim}
The above leads to a successful execution, as the type information about \texttt{fst} is preserved through the type parameter. The parameter \texttt{T} is unbounded, and ranges over all possible types.

Bounded quantification allows us to specify type parameters that range over a subset of all types. Generally, this set is given by all subtypes of a given type.

\begin{verbatim}
  class Pair[T: Stringable box] is Stringable
    var fst: T 
    var snd: T 
  
    new create(fst': T, snd': T) =>
      this.fst = consume fst'
      this.snd = consume snd'
    
    fun string() : String iso^ =>
      this.fst.string() + this.snd.string()
\end{verbatim}
Here \texttt{T} can be any subtype of \texttt{Stringable}.

\subsection{F-bounded Polymorphism}

Recursively bounded quantification, also known as f-bounded quantification, allows type parameters to appear in their own bounds.
\begin{verbatim}
  interface Equatable[A: Equatable[A] #read]
    fun eq(that: box->A): Bool => this is that
\end{verbatim}
In the above, the type parameter \texttt{A} is required to be a subtype of \texttt{Equatable[A]}. This means, if \texttt{String} were to implement \texttt{Equatable[String]}, it would require a method \texttt{fun eq(that: String box): Bool} (which would hopefully be more useful than the default implementation). This means that \texttt{Strings} are comparable to other \texttt{Strings} but not to, say, \texttt{Integers}.

\subsection{Issues with F-Bounded Polymorphism}

F-bounded polymorphism introduces type-checking complexities. It requires that inheritance to be recursive; \texttt{String} implements \texttt{Comparable<String>}, meaning it is defined in terms of its inherited type. When recursive inheritance is mixed with variance in type parameters, polymorphic subtyping becomes undecidable \cite{Kennedy2008}. However, this does not apply in Pony, due to type invariance relative to type parameters. 

F-bounded polymorphism can lead to significant code complexity.
\begin{verbatim}
    [Example to demonstrate this]
\end{verbatim}


\section{Dependent Types}

A dependent type is one defined based upon a value. For example, a "pair of integers" is a type, and a "pair of integers where the first value is greater than the second" is a dependent type. Pony does not have dependent types, but an example, adapted from \cite{Bove2009}, in Agda, a language that does, is below: 
\begin{verbatim}
    Vec : Set -> Nat -> Set
    Vec A zero = Unit
    Vec A (succ n) = A X Vec A n
\end{verbatim}
The signature of a \texttt{zip} function, which requires both \texttt{Vec} to be of the same length, can be defined as:
\begin{verbatim}
    zip : {A B : Set} -> (n : Nat) ->
        Vec A n -> Vec B n -> Vec (A X B) n
\end{verbatim}

We can draw a parallel with polymorphic types: polymorphic types are parameterised by types, and dependent type are parameterised by values.

\subsection{Path Dependent Types}

Path dependent types are a form of dependent type which depends on the value of an abstract type parameter.
In Scala, a nested type depends on a specific instance of the outer type, rather than the outer type itself.
\begin{verbatim}
  trait Animal { a =>
    type Food
    def eats(food: a.Food): Unit = {}
  }

  trait Grass
  trait Meat

  trait Cow extends Animal with Meat {
    type Food = Grass
  }

  trait Lion extends Animal {
    type Food = Meat
  }
\end{verbatim}
In the definitions above, adapted from \cite{Amin2014}, \texttt{Animal} has an \textit{abstract type member} \texttt{Food}. We can see that \texttt{eats} requires its parameter to be of type \texttt{a.Food}, which is a \textit{path-dependent type}. We therefore define \texttt{Cow} to be an \texttt{Animal} that eats \texttt{Grass}, and \texttt{Lion} to be an \texttt{Animal} that eats \texttt{Meat}.
\begin{verbatim}
  val leo = new Lion {}
  val milka = new Cow {}
  val daisy = new Cow {}

  leo.eats(milka)     //ok
  milka.eats(daisy)   //error
\end{verbatim}
Whilst \texttt{leo} can eat \texttt{milka}, since \texttt{leo.eats()} expects type \texttt{Meat} which \texttt{milka} is, \texttt{milka} cannot eat \texttt{daisy} as she isn't \texttt{Grass}.

\section{Soundness}

A sound type system is one in which well-typed programs cannot lead to type errors \cite{Wright1994}. We can prove the type soundness of a system using \textit{subject reduction} and \textit{progress} \cite{Igarashi2001}.

\subsubsection{Subject Reduction}

A type system with subject reduction says that the evaluation of expressions does not cause their type to change. This can be expressed formally with they type rule:

\begin{mathpar}
\inferrule{\Gamma \vdash e_1 : \tau \\ e_1 \to e_2}{\Gamma \vdash e_2 : \tau}
\end{mathpar}

\subsubsection{Progress}

Progress indicates that the evaluation of a program cannot become \textit{stuck}. This means that each program can either be evaluated infinitely, or will reach a state in which it cannot be evaluated further, that is, a value or a type error; the program never gets into an undefined state where no further transitions are possible.

\subsubsection{Type Soundness}

By combining the above mechanisms, if we can show that a well-typed program \footnote{We consider the entire program to be an expression for the definitions above.} reduces to only other well-typed programs, and will never become stuck, we have shown that the system is type sound.

\subsection{Proving Type Soundness in the Presence of Generics}

To prove that a generic type system is sound given a proof of soundness for a non-generic version for the same system, it is sufficient to show that any valid type instantiation on a generic declaration yields a valid concrete declaration \cite{Gordon2012} .

\section{Concurrent Programs} \label{conrules}

\section{Actor Based Models}

Actor based languages \cite{Hewitt1973} provide a simpler model of concurrency than traditional object-oriented languages; instead of relying on mutual exclusion primitives to control threads' access to shared memory, \textit{actors} have their own private state which only they can modify, and rely on message passing to interact with other actors. Actors process messages sequentially, but can execute concurrently. This message passing system eliminates the need for concurrency primitives such as locks and semaphores, and grants data race freedom.

\subsection{Capabilities}

In languages where the programmer can use pointers to reference objects in the heap, a number of potential problems can occur; the aliasing and sharing of mutable state can lead to unintended behaviour.

A solution to this problem is \textit{capabilities}. In the system proposed by \cite{Boyland2001}, capabilities consist of a pointer to an object, and a set of access rights that detail how the pointer can be used. \cite{Boyland2001} introduces the concepts of \textit{base rights} and \textit{exclusive rights}; base rights say a reference can perform an operation, such as writing, on an object, and exclusive rights say that no other reference can perform that operation on the object\footnote{This does not necessarily mean the reference can perform the operation, it would still require the base right to do so.}. This leads to the concept of \textit{compatibility} -- two capabilities on the same object are compatible if their access rights permit the other to exist. Corresponding base and exclusive rights are incompatible as asserting one denies other references of the other.

\subsubsection{Capabilities in Actor Based Models}

We can use capabilities in actor based models to ensure data-race free programs. By ensuring that all references to an object have compatible capabilities, which can be ensured by the type system statically, we know that no two actors can access the same memory in a way that violates the rules described in \ref{conrules}.

Capabilities in actor models give actors the ability to read, mutate, and send a reference safely. Only one actor may possess a mutable reference to an object at a time, and so to send a mutable reference, the sending actor must give up its reference to the object \cite{Srinivasana}.



\subsubsection{Isolation} \label{isolation}

An isolated reference is one where all paths to non-immutable objects reachable from the reference pass through it \cite{Gordon2012}. These isolated references form static regions where mutable objects reachable from the reference can only be reached through the reference, and immutable objects must be either only reachable through the reference of globally immutable \cite{Clebsch2015}.

We can \textit{recover} mutable types back to an isolated state if we can ensure it is an isolated reference. The resulting alias makes stronger local guarantees than the previous mutable alias.

\subsection{Deny Capabilities}

While earlier capability systems \cite{Srinivasana, Gordon2012, Boyland2001} detail what operations are allowed on an object, an other approach is to use capabilities to detail what aliases are \textit{denied} by the existence of a reference \cite{Clebsch2015}. This system of deny capabilities is what we will use to frame the rest of our exploration.

\subsubsection{Aliases}

\textbf{Local aliases} are other references in the same actor to an object that the actor already has a reference to. \textbf{Global aliases} are references to the object that are held by other actors. Deny capabilities describe what a local and global alias is forbidden to do.

A reference is mutable if it denies global read and write aliases, as no other actor can concurrently read or write to the object. Immutability is given by denying global write aliases, so a data race where an actor is reading an object as another is writing to it cannot occur. Denying no glabal aliases makes a reference \textit{opaque}, meaning it can be neither read nor written to.

A reference can be safely sent to another actor if its global and local deny properties are the same, i.e. it does not matter which actor is holding the reference.

A full description of the deny capabilities is given in Table \ref{table:cap}.

\begin{table}[h]
\centering
\begin{tabular}{l|l|l|l|l}
\cline{2-4}
                                                                                               & \begin{tabular}[c]{@{}l@{}}Deny global read/ \\ write aliases\end{tabular} & \begin{tabular}[c]{@{}l@{}}Deny global \\ write aliases\end{tabular} & \begin{tabular}[c]{@{}l@{}}Allow all \\ global aliases\end{tabular} &  \\ \cline{1-4}
\multicolumn{1}{|l|}{\begin{tabular}[c]{@{}l@{}}Deny local read/\\ write aliases\end{tabular}} & \textit{Isolated (iso)}                                                   &                                                                      &                                                                     &  \\ \cline{1-4}
\multicolumn{1}{|l|}{\begin{tabular}[c]{@{}l@{}}Deny local \\ write aliases\end{tabular}}      & Transition (\texttt{trn})                                                          & \cellcolor[HTML]{C0C0C0}\textit{Value (val)}                         &                                                                     &  \\ \cline{1-4}
\multicolumn{1}{|l|}{\begin{tabular}[c]{@{}l@{}}Allow all \\ local aliases\end{tabular}}       & Reference (\texttt{ref})                                                           & \cellcolor[HTML]{C0C0C0}Box (\texttt{box})                                    & \cellcolor[HTML]{656565}\textit{Tag (tag)}                          &  \\ \cline{1-4}
                                                                                               & (Mutable)                                                                 & \cellcolor[HTML]{C0C0C0}(Immutable)                                  & \cellcolor[HTML]{656565}(Opaque)                                    &  \\ \cline{2-4}
\end{tabular}
\caption{Capability matrix, reproduced from \cite{Clebsch2015}}
\label{table:cap}
\end{table}

\subsection{View Point Adaptation}

\begin{figure}[h]
    \centering
    
\begin{tikzpicture}[auto,node distance=1.5cm]
  \node[entity] (node1) {$\alpha$};
  \node[entity] (node2) [below = of node1] {$\iota_1$};
  \node[entity] (node3) [below = of node2]	{$\iota_2$};
  \draw[->] (node1)
    edge	 node {$x, \kappa$}    (node2);
  \draw[->] (node2)
    edge	 node {$f, \kappa'$}	(node3);
  \draw[dashed,->] (node1)
    edge[bend left=90]   node {$x.f, \kappa \triangleright \kappa'$} (node3);  
\end{tikzpicture}

    \caption{Viewpoint Adaptation}
    \label{fig:viewpoint}
\end{figure}

When reading a field from an object, as seen in Figure \ref{fig:viewpoint}, we create a temporary alias whose capability depends on the capabilities of the object and its field. We call this \textit{viewpoint adaptation}, denoted by $\triangleright$ and it can intuitively be thought of as "if \textit{$a_1$} sees \textit{$a_2$} as $\kappa$ and \textit{$a_2$} sees \textit{$a_3$} as $\kappa'$, then \textit{$a_1$} sees \textit{$a_3$} as  $\kappa \triangleright \kappa'$ \cite{Cunningham2008}. This alias is \textit{temporary}, and does not outlive the execution of the expression which created it, such as field read \cite{Clebsch2015}.

\subsection{Extracting Adaptation} \label{extracting}

\begin{figure}[h]
    \centering
    \begin{tikzpicture}[auto,node distance=1.5cm]
  \node[entity] (node1) {$\alpha$};
  \node[entity] (node2) [below = of node1] {$\iota_1$};
  \node[entity] (node3) [below = of node2]	{$\iota_2$};
  \draw[->] (node1)
    edge	 node {$x, \kappa$}    (node2);
  \draw[->] (node2)
    edge	 node{$f, \kappa'$}    (node3);
  \draw[dashed,->] (node1)
    edge[bend left=90]   node {$x.f, \kappa \extract \kappa'$} (node3);  
\end{tikzpicture}
    \caption{Extracting Adaptation}
    \label{fig:extracting}
\end{figure}

The extracting adaptation introduced by Steed \cite{Steed2016} considers the case in Figure \ref{fig:extracting}, where an actor $\alpha$ has an object $x$, which in turn has a field $f$. If $f$ is overwritten with a new value, the return value of the operation is a temporary reference to the old value of the field, whose reference capability is given by $\kappa \extract \kappa'$.

\subsection{Aliasing and Unaliasing}

\subsubsection{Temporary References}

Here, we distinguish between stable references, such as object fields and local variables, and temporary references, such as those created by the adaptations described above.

Considering the capabilities of temporary aliases, we are presented with two additional scenarios which affect unique aliases:

\begin{itemize}
    \item zero stable references exist to an object anywhere in the program. This is similar to the isolation described in Section \ref{isolation}, but its sole reference is temporary (it has one reference removed). This is represented by an additional deny capability \texttt{iso$^-$}.
    \item zero stable \textit{mutable} references exist to an object anywhere in the program. This is similar to the transition deny capability \texttt{trn}, but with an alias removed, and we denote it using \texttt{trn$^-$}.
\end{itemize}

We refer to \texttt{iso$^-$} and \texttt{trn$^-$} as ephemeral capabilities.

\subsubsection{Aliasing}

Aliasing occurs when a new stable reference is created from an existing one. The capability of the new alias depends on the capability of the reference it was created from, and cannot deny more than the original capability.

Any capability that is locally compatible with itself aliases as itself. An isolated reference doesn't allow mutable nor immutable references, and so aliases as an opaque reference. The deny capability \texttt{trn} only allows mutable local aliases, and must alias as a capability that allows a local mutable alias (the initial \texttt{trn}). Thus \texttt{trn} aliases as \texttt{box}.

The ephemeral capabilities described above are similar to their stable counterparts, but with one reference removed. Aliasing them simply adds that reference, and thus they alias as their stable counterparts.

\subsubsection{Destructive Reads and Unaliasing}

An operation which discards an alias, such as the extracting write described in Section \ref{extracting}, removes a stable reference to that object and returns an ephemeral reference to the object.

\section{Pony}

Pony is an open-source, object-oriented, actor-model, capabilities-secure, high-performance programming language that employees the strategy of deny capabilities described in the above sections to ensure static data-race freedom \cite{PonyWeb}.

\subsection{Pony Types}

Types in Pony differ from those in other languages as, in addition to a set of values, they contain a reference capability.

\subsection{Polymorphism in Pony}

Pony supports all the forms of polymorphism detailed in Section \ref{polymorphism}, but with some restrictions introduced by reference capabilities.

\subsubsection{Capability Subtyping}

When thinking about substitutability of types, we must think about how capabilities can be related to one another.

By considering some of the deny capabilities, we can see how they can be substituted for one another; \texttt{iso} is read and write unique, and \texttt{trn} only guarantees write uniqueness, so it is clear that we can substituted an \texttt{iso} for a \texttt{trn}. 

The deny capability type system \cite{Clebsch2015} allows us to build the whole subcapability relation, shown in Figure \ref{fig:subcap}.

\begin{figure}[h]
    \centering
    \begin{tikzpicture}[auto,node distance=1.5cm]
  \node[entity] (iso) {\texttt{iso}};
  \node[entity] (trn) [right = of iso] {\texttt{trn}};
  \node[entity] (ref) [above right = of trn]{\texttt{ref}};
  \node[entity] (val) [below right = of trn]{\texttt{val}};
  \node[entity] (box) [above right = of val]{\texttt{box}};
  \node[entity] (tag) [right = of box]{\texttt{tag}};
  \draw[->] (iso) edge (trn);
  \draw[->] (trn) edge (ref);
  \draw[->] (trn) edge (val);
  \draw[->] (ref) edge (box);
  \draw[->] (val) edge (box);
  \draw[->] (box) edge (tag);
\end{tikzpicture}
    \caption{Subcapability relation}
    \label{fig:subcap}
\end{figure}

\subsubsection{Capability Constraints}

When instantiating a type variable, Pony requires that the capability of the parameter matches the capability of the bound exactly, and is not a subcapability. If a capability is not supplied with the type bound, the method or class must work for all possible capabilities. 

Instead of a specific capability, Pony allows you to specify \textit{capability constraints} in type bounds, which allows any of a set of concrete capabilities. The full list of capability constraints is detailed in Table \ref{table:constraints}.

\begin{table}[h]
    \centering
    \begin{tabular}{|l|l|l|}
\hline
\textbf{Constraint} & \textbf{Allowed Capabilities} & \textbf{Description}                                                                                       \\ \hline
\texttt{\#any}               & \texttt{iso, trn, ref, val, box, tag}  & Default constraint                                                                                         \\ \hline
\texttt{\#read}              & \texttt{ref, val, box}                 & Anything that can be read from                                                                             \\ \hline
\texttt{\#share}             & \texttt{val, tag}                      & \begin{tabular}[c]{@{}l@{}}Anything that can be sent to\\ more than one actor\end{tabular}                 \\ \hline
\texttt{\#send}              & \texttt{iso, val, tag}                 & \begin{tabular}[c]{@{}l@{}}Anything that can be sent to\\ an actor\end{tabular}                            \\ \hline
\texttt{\#alias}             & \texttt{ref, val, box, tag}            & \begin{tabular}[c]{@{}l@{}}Set of capabilities that alias as \\ themselves (used by compiler)\end{tabular} \\ \hline
\end{tabular}
    \caption{Capability Constraints, reproduced from \cite{PonyTut}}
    \label{table:constraints}
\end{table}

\subsubsection{Variance in Pony Generic Parameters}

In Pony, type parameters are invariant: a \texttt{List[Any box]} is neither a subtype nor a supertype of a \texttt{List[String box]}. This is due to the fact that a type parameter may be used both covariantly (as a return type) and contravariantly (as a parameter type) in the same method \cite{Igarashi2001}.

\subsection{Pony Formal Models}

There have been several formal models for Pony proposed since its creation. The first, introduced in \cite{Clebsch2015} showed that the Pony type system guarantees freedom from data-races but lacked a number of important language features. \textit{Pony}$^{GS}$ \cite{Steed2016} then formalised a larger subset of Pony, simplifying and enhancing it. However, it still missed a number of features required for a complete formal model, such as recovery and generics. Most recently, \textit{Pony}$^{PL}$ \cite{Lietar2017} introduced a formal model incorporating generics, involving rules for translation to a non-generic model \textit{Pony}$^0$.