\chapter{Introduction}

\section{Motivations}

There is inherent complexity in writing concurrent applications; care must be taken to prevent concurrent operations from interfering with each others' data as it could potentially create data races. The widely used solution this problem is explicit concurrency management through concurrency primitives. These require programmers to be able to reason about how threads will interact with each other, and to be constantly mitigating against concurrency issues. Despite this additional burden on programmers, they are exposed to risks such as deadlocking.

Pony is an actor-model language that allows developers to write high-performance concurrent applications that are guaranteed to be data-race free. It uses a system of reference capability to enforce this at runtime; if the program compiles, it is data-race free.

A number of formal models for Pony have been created to verify various aspects of its design. These models, collectively, guarantee a number of features of the language, but a single unified model does not yet exist. Additionally, the current model of generics supports F-bounded polymorphism, an alternative to which exists in the form of Materials/Shapes.

\section{Objectives}

We want to design a formal model of generics as they appear in Pony and investigate the replacement of its support of F-bounded polymorphism by adapting a novel, simpler and almost as powerful approach, called Materials/Shapes.

To this end, we aim to create a unified formal model that builds on the achievements of previous efforts to formalize Pony, and create the most encompassing model to date.

We will then evaluate F-bounded polymorphism against Materials/Shapes and integrate a model of generics into the formal model we have created. If we can prove this model sound, we can then evaluate integrating this model into the Pony compiler.

%\section{Contributions}